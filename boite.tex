\documentclass{article}

\usepackage[utf8]{inputenc}
\usepackage[T1]{fontenc}
\usepackage[french]{babel}
\usepackage[top=3cm, bottom=3cm, left=3cm, right=3cm]{geometry}
\usepackage{tikz}
\usepackage{pgfplots}

\usetikzlibrary{calc}

\title{Boite à moustache}
\author{Chabassier Luc}

\begin{document}
\maketitle

Valeurs : \\
\begin{center}
	\begin{tabular}{|c|c|}
		\hline
		\textbf{Xi} & \textbf{Ni} \\
		\hline
		5 & 47 \\
		\hline
		15 & 85 \\
		\hline
		25 & 101 \\
		\hline
		40 & 142 \\
		\hline
		65 & 84 \\
		\hline
		100 & 31 \\
		\hline
	\end{tabular}
\end{center}

% On considère une boite telle que min = 0, q1 = 25, q2 = 30, q3 = 50, max = 100
Boite à moustaches : \\
\begin{center}
	\begin{tikzpicture}
		% Les paramètres
		\pgfmathsetmacro\min{10};
		\pgfmathsetmacro\qone{20};
		\pgfmathsetmacro\qtwo{25};
		\pgfmathsetmacro\qthree{40};
		\pgfmathsetmacro\max{60};
		\pgfmathsetmacro\scale{0.2};
		\pgfmathsetmacro\offset{5};

		% Valeurs pour l'affichage de ces nombres
		\pgfmathtruncatemacro\vmin{\min};
		\pgfmathtruncatemacro\vqone{\qone};
		\pgfmathtruncatemacro\vqtwo{\qtwo};
		\pgfmathtruncatemacro\vqthree{\qthree};
		\pgfmathtruncatemacro\vmax{\max};

		% Les valeurs utiles
		\pgfmathsetmacro\minp{\min * \scale};
		\pgfmathsetmacro\qonep{\qone * \scale};
		\pgfmathsetmacro\qtwop{\qtwo * \scale};
		\pgfmathsetmacro\qthreep{\qthree * \scale};
		\pgfmathsetmacro\maxp{\max * \scale};

		% La boite
		\draw[very thick] (\minp,0.5) -- (\qonep,0.5);
		\draw[very thick] (\qthreep,0.5) -- (\maxp,0.5);
		\draw[very thick] (\qonep,0) -- (\qthreep,0);
		\draw[very thick] (\qonep,1) -- (\qthreep,1);
		\draw[very thick] (\minp,0) -- (\minp,1) node[red,above] {\vmin};
		\draw[very thick] (\qonep,0) -- (\qonep,1) node[red,above] {\vqone};
		\draw[very thick] (\qtwop,0) -- (\qtwop,1) node[red,above] {\vqtwo};
		\draw[very thick] (\qthreep,0) -- (\qthreep,1) node[red,above] {\vqthree};
		\draw[very thick] (\maxp,0) -- (\maxp,1) node[red,above] {\vmax};

		% Les aides à visualiser
		\draw[red,thick] (\minp,-1) -- (\minp,0);
		\draw[red,thick] (\qonep,-1) -- (\qonep,0);
		\draw[red,thick] (\qtwop,-1) -- (\qtwop,0);
		\draw[red,thick] (\qthreep,-1) -- (\qthreep,0);
		\draw[red,thick] (\maxp,-1) -- (\maxp,0);

		% L'axe
		\pgfmathsetmacro\mina{\minp - \offset * \scale};
		\pgfmathsetmacro\maxa{\maxp + \offset * \scale};
		\draw[->] (\mina,-1) -- (\maxa,-1);

		\pgfmathsetmacro\offp{\minp + \scale * \offset}
		\foreach \x in {\minp,\offp,...,\maxp} {
			\pgfmathtruncatemacro\abs{(\x / \scale) + 0.1} % Le + 0.1 est pour absorber les imprécisions
			\draw (\x,-0.9) -- (\x, -1.1)
				node[below] {\abs};
		}
	\end{tikzpicture}
\end{center}
\end{document}
